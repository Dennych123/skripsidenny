\chapter{PENDAHULUAN}

\section{Latar Belakang}

Industri otomotif global saat ini sedang menghadapi disrupsi besar seiring gencarnya transformasi digital dan tren kendaraan listrik. Kondisi ini ditandai dengan transformasi digital yang secara drastis mengubah industri otomotif dan mengganggu model bisnis tradisional, sehingga peluang bisnis baru terkait Industri 4.0 pun bermunculan dan perusahaan dituntut beradaptasi dengan lingkungan baru \parencite{llopis2020industry}. Pengenalan kendaraan listrik secara bertahap ke pasar global juga membawa efek disrupsi signifikan. Dampak ini diperkuat oleh kebijakan pengurangan emisi gas rumah kaca dan pemanfaatan sumber energi terbarukan, sehingga industri otomotif dipaksa menyesuaikan diri dengan regulasi baru terkait efisiensi energi dan keberlanjutan . Fenomena transformasi digital dan elektrifikasi kendaraan tersebut menuntut produsen komponen otomotif meningkatkan efisiensi produksi serta berinovasi dalam pengendalian proses manufaktur untuk mempertahankan daya saing di pasar global.

PT~XYZ sebagai perusahaan manufaktur komponen otomotif nasional tidak lepas dari tekanan tersebut. Bersaing di pasar global, PT~XYZ dituntut meningkatkan efisiensi proses produksi serta beradaptasi dengan cepat terhadap teknologi mutakhir. Untuk mencapai hal ini, perusahaan menekankan penerapan konsep Industri~4.0 dalam proses produksinya, antara lain melalui otomatisasi cerdas dan digitalisasi mesin produksi. Dengan menghubungkan sistem mesin melalui teknologi digital dan Internet of Things (IoT), PT~XYZ berharap dapat meningkatkan produktivitas dan fleksibilitas operasional, sehingga lebih siap menghadapi perubahan tuntutan pasar otomotif akibat tren kendaraan listrik.

Di lapangan, penerapan strategi digitalisasi di PT~XYZ didorong oleh temuan bahwa sebagian besar mesin produksi sudah menggunakan teknologi kontrol canggih. Berdasarkan data internal PT~XYZ, sekitar 91\% mesin di Departemen \textit{Machinery (Special Purpose Machine Maker)} menggunakan \textit{Programmable Logic Controller} (PLC) merek Omron dengan protokol komunikasi FINS (Factory Interface Network Service). Kondisi ini menegaskan perlunya pengembangan sistem \textit{Human-Machine Interface} (HMI) berbasis web yang kompatibel dengan PLC dan protokol tersebut.

Namun, penggunaan sistem SCADA komersial yang tertutup berpotensi menimbulkan ketergantungan pada vendor tertentu (\textit{vendor lock-in}). Ketergantungan ini dapat menimbulkan risiko biaya tambahan ketika perangkat keras yang digunakan mengalami disfungsi, dihentikan produksinya (discontinue), atau tidak lagi didukung secara penuh oleh vendor. Perusahaan perlu mengeluarkan biaya besar untuk penggantian modul PLC, upgrade lisensi SCADA, atau penyesuaian perangkat baru yang kompatibel \parencite{banthia2024vendorlockin}. Risiko ini semakin nyata pada sistem tertutup yang tidak menyediakan fleksibilitas integrasi dengan teknologi masa depan.

Oleh karena itu, pengembangan sistem SCADA berbasis web sumber terbuka seperti FUXA menjadi solusi strategis. Dengan menggunakan platform open source seperti FUXA, perusahaan dapat membangun visualisasi proses produksi modern dan mengendalikan mesin-mesin produksi secara \textit{real-time} melalui antarmuka web yang fleksibel, tanpa terkendala biaya lisensi tinggi maupun batasan teknologi dari vendor tertentu. Pendekatan ini diharapkan mendukung upaya monitoring dan pengendalian produksi PT~XYZ dalam rangka efisiensi dan adaptasi teknologi secara menyeluruh.


\section{Rumusan Masalah}
Rumusan masalah dalam penelitian ini adalah sebagai berikut:
\begin{enumerate}
    \item Bagaimana menambahkan dukungan protokol FINS pada FUXA dengan fitur yang setara dengan protokol lain seperti Modbus?
    \item Bagaimana memastikan proses pembacaan dan penulisan data melalui protokol FINS berjalan stabil serta dapat ditampilkan dengan benar pada antarmuka pengguna (UI)?
\end{enumerate}

\section{Tujuan dan Manfaat Penelitian}
\subsection{Tujuan Penelitian}
Penelitian ini bertujuan untuk mengimplementasikan dukungan protokol FINS pada FUXA, meliputi:
\begin{itemize}
    \item Pengambilan data secara berkala (polling),
    \item Penulisan nilai ke PLC,
    \item Dukungan \textit{Data Acquisition} (DAQ),
    \item Integrasi sistem alarm berbasis tag.
\end{itemize}

\subsection{Manfaat Penelitian}
Manfaat dari penelitian ini antara lain:
\begin{itemize}
    \item Menyediakan solusi SCADA sumber terbuka yang kompatibel dengan PLC Omron, khususnya untuk industri kecil dan menengah.
    \item Memberikan kontribusi nyata dalam pengembangan perangkat lunak sumber terbuka di bidang otomasi industri.
\end{itemize}

\section{Ruang Lingkup Penelitian}
Penelitian ini memiliki ruang lingkup sebagai berikut:
\begin{itemize}
    \item Fokus pada implementasi komunikasi FINS melalui protokol UDP/TCP.
    \item Penggunaan area memori standar PLC Omron seperti \texttt{DM}, \texttt{CIO}, \texttt{W}, dan sejenisnya.
    \item Pengembangan terbatas pada pembacaan dan penulisan tag serta visualisasi nilai melalui UI FUXA.
    \item Pengujian dilakukan menggunakan perangkat lunak analisis jaringan seperti Wireshark serta PLC fisik maupun simulator.
\end{itemize}

\section{Hipotesis Penelitian}
Hipotesis dari penelitian ini adalah:
\begin{itemize}
    \item Jika protokol FINS berhasil diintegrasikan ke dalam FUXA, maka sistem SCADA tersebut akan mampu membaca dan menulis data dari PLC Omron secara stabil dan akurat.
    \item Performa polling dan DAQ yang dihasilkan akan setara dengan protokol komunikasi lain seperti Modbus.
\end{itemize}

\section{Sistematika Penulisan}
Adapun sistematika penulisan dalam laporan ini adalah sebagai berikut:
\begin{itemize}
    \item \textbf{Bab 1} – Pendahuluan: berisi latar belakang, rumusan masalah, tujuan dan manfaat, ruang lingkup, hipotesis, dan sistematika penulisan.
    \item \textbf{Bab 2} – Tinjauan Referensi: membahas teori dan referensi terkait, seperti protokol FINS, FUXA, dan komunikasi industri.
    \item \textbf{Bab 3} – Metodologi Penelitian: menjelaskan tahapan dan metode penelitian yang digunakan.
    \item \textbf{Bab 4} – Implementasi dan Pengujian: menyajikan proses integrasi FINS pada FUXA serta hasil pengujian.
    \item \textbf{Bab 5} – Kesimpulan dan Saran: berisi simpulan dari hasil penelitian serta rekomendasi untuk pengembangan lebih lanjut.
\end{itemize}
