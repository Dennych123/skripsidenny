\chapter{METODOLOGI PENELITIAN}

\section{Jenis Penelitian}
Jenis penelitian ini adalah penelitian rekayasa perangkat lunak (software engineering research), yang bertujuan untuk merancang, mengimplementasikan, dan mengevaluasi penambahan dukungan protokol FINS (Factory Interface Network Service) pada sistem SCADA open-source FUXA. Penelitian ini bersifat terapan dengan pendekatan kuantitatif dan kualitatif dalam menguji fungsionalitas dan performa sistem.

\section{Metode Pengembangan Sistem}
Metode yang digunakan dalam pengembangan perangkat lunak ini adalah metode iteratif dan inkremental, dengan tahapan sebagai berikut:

\begin{enumerate}
    \item \textbf{Analisis Kebutuhan:} Mengidentifikasi kebutuhan pengguna terhadap integrasi protokol FINS pada FUXA, termasuk dukungan konfigurasi parameter (DA1, SA1, Unit Address), polling tag, penulisan nilai ke PLC, serta fitur DAQ dan alarm.
    \item \textbf{Perancangan Sistem:} Mendesain struktur konektor FINS di sisi server dan antarmuka pengguna di sisi client, sesuai dengan arsitektur FUXA (Node.js dan Angular).
    \item \textbf{Implementasi:} Mengembangkan modul konektor FINS (server/runtime/devices/fins), komponen konfigurasi tag dan perangkat (Angular), serta logika polling dan penulisan data.
    \item \textbf{Pengujian dan Evaluasi:} Melakukan pengujian fungsional, integrasi, serta monitoring paket FINS menggunakan Wireshark untuk memastikan komunikasi berjalan benar.
    \item \textbf{Perbaikan dan Optimalisasi:} Menangani error, optimasi performa polling, memory leak (EventEmitter), serta penambahan fitur lanjutan (alarm, DAQ).
\end{enumerate}

\section{Alat dan Bahan}
\begin{itemize}
    \item \textbf{Perangkat Keras:} Laptop/PC, jaringan LAN, PLC Omron (atau simulator).
    \item \textbf{Perangkat Lunak:} 
    \begin{itemize}
        \item FUXA (https://github.com/frangoteam/FUXA)
        \item Node.js, Angular CLI, Git
        \item Wireshark untuk sniffing paket FINS
        \item Visual Studio Code untuk pengembangan
    \end{itemize}
    \item \textbf{Library Tambahan:} 
    \begin{itemize}
        \item \texttt{node-fins} atau modifikasi client FINS custom
        \item Angular Material untuk UI komponen konfigurasi
    \end{itemize}
\end{itemize}

\section{Tahapan Penelitian}
Penelitian dilakukan dalam beberapa tahap berikut:

\begin{table}[H]
\centering
\begin{tabular}{|c|p{8cm}|}
\hline
\textbf{No} & \textbf{Kegiatan} \\
\hline
1 & Studi literatur tentang protokol FINS, SCADA, dan arsitektur FUXA \\
2 & Perancangan konektor dan struktur konfigurasi perangkat/tag \\
3 & Implementasi modul konektor dan antarmuka pengguna \\
4 & Pengujian komunikasi, pengamatan melalui Wireshark \\
5 & Evaluasi dan dokumentasi hasil integrasi \\
\hline
\end{tabular}
\caption{Tahapan Pelaksanaan Penelitian}
\end{table}

\section{Metode Pengujian}
Pengujian dilakukan secara black-box dan white-box:

\begin{itemize}
    \item \textbf{Fungsional:} Pengujian konfigurasi parameter, pembacaan dan penulisan tag.
    \item \textbf{Integrasi:} Validasi konektivitas antar komponen client-server.
    \item \textbf{Monitoring:} Analisis lalu lintas FINS menggunakan Wireshark.
    \item \textbf{Stabilitas:} Observasi memory leak, error event listener, dan retry mechanism.
\end{itemize}
